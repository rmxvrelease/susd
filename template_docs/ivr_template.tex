\documentclass{report}
\usepackage{longtable}
\usepackage[margin=1.0cm]{geometry}

\title{Laudo quantitativo}
\author{Rafael Monteiro}
\date{\today}


\begin{document}

\maketitle

\section{Conclusões}
Tem que ver isso aí

\section{Resumo total}
\begin{tabular}
	\caption{Resumo do período completo}
\end{tabular}

\section{Resumo mensal}
Explicar como a correção foi efetuada

\begin{longtable}[c]{|p{1cm}|p{2.3cm}|p{2.3cm}|p{2.3cm}|c|c|}
	\caption{Resumo mês a mês} \\ \hline
	\textbf{Mês} &
	\textbf{Valor devido IVR (SIA)} &
	\textbf{Valor devido IVR (SIH)} &
	\textbf{Valor devido IVR (SIH + SIA)} &
	\textbf{IVR corrigido} &
	\textbf{Indice de correção} 
	\endhead \hline

	jul-18 & 10.0 & 20.0 & 30.0 & 60.0 & 100\% \\ \hline
	jul-18 & 10.0 & 20.0 & 30.0 & 60.0 & 100\% \\ \hline
	jul-18 & 10.0 & 20.0 & 30.0 & 60.0 & 100\% \\ \hline
	jul-18 & 10.0 & 20.0 & 30.0 & 60.0 & 100\% \\ \hline
\end{longtable}

	
\end{document}
